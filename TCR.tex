\documentclass[
	a4paper,
	landscape,
	%twoside,
	10pt,
	article
]{article}
\usepackage[
	a4paper,
	landscape,
	twocolumn,
	left=0.8cm,
	right=0.3cm,
% Weird top and bottom margins because of fancyhdr package
	top=1.8cm,
	bottom=-0.3cm,
	columnsep=1cm,
% Set margins on even and odd pages equal
	hmarginratio=1:1,
	asymmetric
]{geometry}
\usepackage[english]{babel}
\usepackage[utf8]{inputenc}
\usepackage{textcomp}
\usepackage{amsmath}
\usepackage{amsfonts}
\usepackage{graphicx}
\usepackage{float}
\usepackage{listings}
\usepackage{color}
\usepackage[colorinlistoftodos]{todonotes}
\usepackage[compact]{titlesec}		% shrink section whitespace
\usepackage{ifthen}
\usepackage{nicefrac}
\usepackage{hyperref}

%*************** Layout ***************
\setlength{\columnseprule}{0.2pt}
\newcommand{\latexcolumnseprulecolor}{\color{red}}
\titlespacing{\section}{0pt}{0pt}{0pt}
\sloppy

% A red divider in the middle of the page
\usepackage{etoolbox}
\makeatletter
\patchcmd\@outputdblcol{% find
	\normalcolor\vrule
}{% and replace by
	\latexcolumnseprulecolor\vrule
}{% success
}{% failure
	\@latex@warning{Patching \string\@outputdblcol\space failed}%
}
\makeatother


%*************** Title ***************

% all the \vspace are for reducing the vertical spacing
\title{
	\vspace{-5em}
	\includegraphics[scale=0.7]{./logo.eps}\\
	\vspace{-1.5em}
	Team Code Reference
	\vspace{-0.7em}
}
%\author{ Jens Heuseveldt, Ludo Pulles  \& Peter Ypma }
\author{
	\Large \textbf{ Jens Heuseveldt, Ludo Pulles  \& Peter Ypma}\\
%	Jens Heuseveldt, Ludo Pulles  \& Peter Ypma
}
\date{
	\vspace{-0.7em}
	NWERC\\
	November 20, 2016
	\vspace{-1.9em}
}

%*************** Table of Contents ***************
\usepackage[toc]{multitoc}			% multicolumn toc
\usepackage{tocloft}				% to reduce toc spacing
\renewcommand*{\multicolumntoc}{2}
% reduce section spacing in toc
\setlength{\cftbeforesecskip}{-1pt}
\setlength{\cftbeforesubsecskip}{-1.5pt}
% remove the toc title
\makeatletter
\renewcommand{\@cftmaketoctitle}{}
\makeatother


%*************** Headings ***************
\usepackage{fancyhdr}
\pagestyle{fancy}
\fancyhead{}
\fancyfoot{}
\setlength{\headsep}{0.4em}
\setlength{\footskip}{0em}

% two sided
%\fancyhead[RE]{\bfseries Git Diff Solution \hspace{6.5em}}
%\fancyhead[LO]{\hspace{5em} Utrecht University}
%\fancyhead[LE,RO]{\thepage}
%\fancyhead[C]{\leftmark}

% one sided
\fancyhead[L]{\hspace{5em} Utrecht University \phantom{-} \bfseries
Git Diff Solution}
\fancyhead[R]{\thepage \hspace{0.5em}}
\fancyhead[C]{\leftmark}

%*************** Code highlighting ***************
\lstset{
	backgroundcolor=\color{white},
	tabsize=4,
	language=C++,
	basicstyle=\footnotesize\ttfamily,
	frame=lines,
	numbers=left,
	numberstyle=\tiny,
	numbersep=5pt,
	breaklines=true,
	keywordstyle=\color[rgb]{0, 0, 1},
	commentstyle=\color[rgb]{0, 0.5, 0},
	stringstyle=\color{red}
}


%*************** Section entries ***************
% \entry{name}{description}{snippet location}{complexity}{dependencies}
\newcommand{\entry}[5]{
	\subsection{#1}
	#2
	\ifthenelse{\equal{#4}{}}{}{\noindent\textbf{Complexity:} #4}
	\ifthenelse{\equal{#5}{}}{}{\noindent\textbf{Dependencies:} #5}
	\ifthenelse{\equal{#3}{}}{}{\lstinputlisting[firstline=2]{#3}}
}
\newcommand{\otherentry}[3]{
	\subsection{#1}
	#2
	\lstinputlisting[language=]{#3}
}


%*************** Begin document ***************
\begin{document}


%*************** Reduce align spacing ***************
\setlength{\abovedisplayskip}{0pt}
\setlength{\belowdisplayskip}{0pt}
\setlength{\abovedisplayshortskip}{0pt}
\setlength{\belowdisplayshortskip}{0pt}

%*************** Titlepage ***************
{\let\newpage\relax\maketitle}
\tableofcontents
\thispagestyle{empty}
\newpage

%*************** Contents ***************
\section{Voorbereiding}
\begin{itemize}
\setlength\itemsep{0em}
\item Goed slapen
\item Genoeg flesjes water mee
\item Toetsenbord inclusief kabel
\item Iets te eten mee (bananen, ligakoeken en dat soort dingen.
\item Pennen, potloden, passer, geodriehoek en markeerstiften mee.
\end{itemize}

\section{Begin van contest}
Peter en Ludo nemen \textbf{op hun gemak} alle opgaven samen door. Daarbij noteren ze het volgende:
\begin{itemize}
\setlength\itemsep{0em}
\item een overzichtslijst van alle opgaven, waarop aangegeven staat van welk type de opgaven zijn en wie de opgave verder gaat uitwerken.
\item Bij opgaven waar dat van toepassing is: noteren welk deel van de TCR daar mogelijk handig voor nodig is. Dit blaadje wordt direct aan Jens doorgegeven.
\end{itemize}

Jens typt vervolgens de volgende dingen op de computer:

\entry{VIM}{}
{./geometry/vim.cpp}{}
{}

\entry{Template}{}
{./geometry/template.cpp}{}
{}

\subsection*{Overig}
Wanneer de VIM-instellingen en de template klaar is, maakt Jens achtereenvolgens:
\begin{itemize}
\setlength\itemsep{0em}
\item De testcase bestanden: A1.in, A2.in, etcetera.
\item Het typen van de delen van de TCR die waarschijnlijk nodig zijn bij de opgaven (hij krijgt deze van Ludo en Peter die deze lijst maken tijdens het doorlopen van de opgaven).
\end{itemize}

In de tussentijd werkt Ludo op papier \'e\'en \`a twee niet zeer gemakkelijke opgaven uit die hij als hij er klaar voor is (of als Jens klaar is, gaat typen). Vanaf dat moment focused Jens zich op de opgaven. waarvan Peter/Ludo dachten dat Jens die goed op kan lossen en de opgaven die heel snel opgelost zijn door andere teams. Peter is ondertussen bezig met de wiskundigere opgaven.

\newpage

\section{Geometry}
\subsubsection*{Tricky testcases}
\begin{itemize}
\setlength\itemsep{0em}
	\item Verticale lijnen
	\item Evenwijdige lijnen
	\item Niet convex
	\item Wat gebeurt er als het convex omhulsul uit alle punten bestaat?
	\item Let op afrondingen (bij floating points)
\end{itemize}


\entry{Punten (2D) en rotaties}{}
{./geometry/points.cpp}{}
{}
\newpage
\entry{Lijnen}{}
{./geometry/lines.cpp}{}
{}
\newpage
\entry{Veelhoeken}{}
{./geometry/polygon.cpp}{}
{}
\entry{Cirkels}{}
{./geometry/circles.cpp}{}
{}

\end{document}